\documentclass[12pt]{article}
 
\usepackage[margin=1in]{geometry} 
\usepackage{amsmath,amsthm,amssymb, float}
\usepackage{listings}

\usepackage[utf8]{inputenc}
\usepackage{amsfonts}
\usepackage{graphicx}
%\usepackage{subcaption} alternative to subfig
\usepackage[round]{natbib}
\usepackage[english]{babel}
\usepackage{marginnote}
\usepackage{array}%for tables
\usepackage{hyperref}
\usepackage{booktabs}
\usepackage{subcaption}
\usepackage{adjustbox}
\usepackage[round]{natbib}
%\usepackage{caption}
%\usepackage{siunitx}
    

%\usepackage{subfig}  %to align two figures side by side
\usepackage{float}
\usepackage[font={footnotesize}]{caption}  %to customize captions
\usepackage[noorphans,font=itshape]{quoting}


\numberwithin{equation}{section} %gets equation numbering follow the section
\linespread{1.3}

\graphicspath{{grafici/}}
\setlength{\parindent}{1cm}
\setlength{\parskip}{0.5cm plus4mm minus3mm}


\AtBeginDocument{\renewcommand{\bibname}{References}} %to call bibliography references
 
\newcommand{\N}{\mathbb{N}}
\newcommand{\Z}{\mathbb{Z}}
 
 
\begin{document}
 
% --------------------------------------------------------------
%                         Start here
% --------------------------------------------------------------
 
\title{ Retail Traders Through Booms and Busts: \\ \vspace{1mm}
 Evidence from Robinhood \\ \vspace{7mm} \small NYU Stern Finance Ph.D. \\ Second Year Paper}
\author{Pietro Reggiani}
\date{\today}
 
\maketitle

\begin{abstract}
Test of the abstract
\end{abstract}

\newpage

\tableofcontents

\newpage


\section{Introduction}

With the arrival of the Covid-19 pandemic, the habits of most people changed drastically. An example is the change in how people approach financial markets, and the amount of attention devoted to them. With the aid of new technologies, barriers to entry for people to invest in financial markets have lowered. The pandemic outbreak forced people in their homes, depriving them of most occupations, leisure, hobbies. With broad access to the internet however, many were able to access financial trading platforms. Thanks to new players in the retail brokerage market, who made it even simpler to open a low cost account and trade, there was a boom in market participation also from people that had no investing experience.

This phenomenon put retail investors under the spotlight, fostering renewed curiosity around their behavior and preferences. The objective of this paper is to study the investment choices of a particular sample of retail investors, tilted towards younger people that invest through an application on their smart-phones. In particular, the goal is to analyze their portfolio decisions before, during and after the market collapse induced by the recent pandemic. Using data from a widespread brokerage platform, we observe the cross-section of popularity of different stocks among investors over three different sample periods: the period before the emergence of Covid-19, until February 2020, the market collapse between February and March 2020, and the subsequent market recovery until August 2020.

The findings suggest that the reaction of these investors during the crisis is quite different from "flight to quality" dynamics. In fact, we find that since the onset of the pandemic investors' interest was directed strongly towards the weakest firms. This is consistent with some sort of contrarian mentality, and a search for value in inexpensive stocks. This can lead to excessive risk taking. Moreover, contrary to other existing research, we find no evidence of a heightened environmental taste on the part of investors since the pandemic.

These results are important because they shed light on a new sample of retail market participants, who are likely to behave differently especially in periods of turmoil as the one we analyze. As the ease of access to financial markets is projected to keep growing due to the adoption of new technologies, it becomes crucial to improve our understanding of the preferences of the new investors that enter markets. This has potential implications also for regulation, which may want to guide or limit risk taking on the part of some segments of the population.

The remainder of the paper is structured as follows. Section \ref{literature} collocates this work in the context of existing literature. Section \ref{data} describes the data set from the brokerage platform, and shows some descriptive statistics. Section \ref{results} presents the main empirical analysis and results. Finally, section \ref{conclusion} concludes.


\section{Related Literature} \label{literature}

This paper relates directly to the vast literature on investor behavior: in particular, the focus is on retail investors. Some of the most prominent work in this field is due to Brad Barber and Terrance Odean. Exploiting detailed data on portfolios and trades from a discount brokerage firm, they studied the behavioral nuances of investors' choices, leading to several influential papers. Their interest is primarily devoted to departures from perfect rationality. For a broad overview of their work, one can read \cite{Barber_2013}.

\cite{Barber_2007} find that individual investors' stock buying behavior is driven by attention: presence in the news, trading volume or extreme daily returns are associated with more purchases. This effect is not present on the sale side, because usually people only sell stocks they already own. A similar patter is observed by \cite{Liaukonyte_2019}, who observe high frequency trading reactions to television advertisements. 

 \cite{Barber_2000} show that individual investors who hold common stock pay a large penalty for active trading. They attribute this effect to overconfidence. As supporting evidence for this explanation, in \cite{Barber_2001}, they document that excessive trading frequency is more common among men, who are known from Psychology to be more prone to overconfidence than women.

An interesting feature of retail investors' preferences -- related to our findings in this paper -- is that they seem to favor stocks that display "lottery-like" characteristics, as shown in \cite{Han_2013}. Bankrupt stocks are an instance of such assets: \cite*{Coelho_2014} find that they attract especially retail traders who use them to gamble in the market. Among our results, we find evidence of a similar behavior in a peculiar subset of retail traders; those active on the \textit{Robinhood} platform. More details on this sub-sample of investors is presented in the next section.

Another important strand of literature connected to this work is the one on investor reaction to crises and in particular to the recent Covid-19 pandemic. Some studies have focused on the impact on stock returns, and on determining which characteristics made assets more resilient in the downturn. \cite{Ramelli_2020} study which were the most important value drivers in the cross-section. \cite*{albuquerque2020} observe that firms with better environmental practices performed better during the market slide, similarly to \cite{Garel_2020}.

Some research has paid attention specifically to the reactions of investors. One example is \cite{pastor2020mutual}, that examines performance and flows of active equity mutual funds during 2020. Their findings suggest that even during the turmoil, active funds performed worse that passive ones. In terms of flows, they find that sustainability ratings and screens were strong predictors of fund inflows.

The work that is most closely related to this one is probably \cite*{Glossner2020}. Their paper studies institutional investors' portfolio adjustments at the onset of the Covid-19 pandemic, using holdings data for different institutions such as mutual funds, investment advisers, pension funds and hedge funds. They find evidence of a general flight to quality on the part of these investors, who reallocated capital towards stocks with strong financials in 2020Q1. They run a similar analysis using the \textit{Robinhood} retail investors, finding evidence of a somewhat opposite behavior: at the beginning of the crash, retail traders favored exactly the weakest firms. In this paper, we will confirm and extend some of their results on the retail sample, especially comparing investor behavior during the market collapse to the one during the subsequent recovery.









\section{Data and Summary Statistics} \label{data}

The analysis focuses on a peculiar sample of retail investors, the users of a trading platform called \textit{Robinhood}\footnote{https://robinhood.com/us/en/.}. This company was the first to offer a commission-free brokerage account, easily accessible to investors through a smart-phone application. 

The platform currently has approximately 13 million active users\footnote{https://www.businessofapps.com/data/robinhood-statistics/.} and \$20 billion of assets under management. Relative to their competitors, the peculiarity of \textit{Robinhood} is that their users tend to be considerably younger, with an average age of 31\footnote{https://www.nytimes.com/2020/07/08/technology/robinhood-risky-trading.html.}, and that the average account size is smaller, between \$1,000 and \$5,000, far from the \$100,000 of the closest competitor. While the fact of considering such a peculiar sub-sample of retail investors is a potential drawback in terms of external validity, it can represent an advantage in terms of novelty. \textit{Robinhood} users are likely to be relatively inexperienced investors, people that were only recently enabled by new technologies to participate in financial markets. Therefore, they were not previously studied. As the adoption of financial technologies becomes increasingly widespread, it becomes more and more important to study the behavior of these investors.


The main data set is provided directly from the platform through an API\footnote{for more details, visit the data source at https://robintrack.net/.}. Approximately every hour, the company provides the number of users currently holding each asset. The hourly figure is then averaged at the daily level. The sample period stretches from May 2018 until August 2020. These data are publicly available, and were previously analyzed in other papers, such as \cite{Stein_2020}, \cite{Welch_2020}, \cite{Pagano_2020} and \cite{Moss_2020}.

It is important to underline the main limitation of the data: the value of investors' positions in each asset is unknown. Intuitively, each investor can own multiple shares of the same stock. In addition, it is possible to invest in fractions of a single share. This means that the numerosity of users is not sufficient to compute portfolio weights. Notice that also the total number of users on the platform is unknown, which prevents us from knowing the fraction of users holding each asset. Therefore, the main variable of interest will be the number of users holding a stock over time. This can be interpreted as a measure of the popularity of that asset among investors.

The available data include stocks and ETFs that users can invest in on the platform. Information on derivative positions is not available. For the purpose of this paper, the universe of assets is limited to stocks and ADRs listed in the three main U.S. exchanges: NYSE, AMEX, and NASDAQ (Compustat exchange codes 11,12,14). Financial firms ($GICS$ code 40) are excluded from the sample.

The Robinhood data are matched to company and security level information from Compustat Capital IQ North America. Company financial statement data are at the yearly frequency, so daily observations for year $t$ are associated to financials from end of year $t-1$. Daily stock prices and returns come from the Compustat Securities Daily database. All the Compustat data are accessed through $WRDS$\footnote{https://wrds-www.wharton.upenn.edu/.}.

Finally, we gather ESG\footnote{I.e., Environmental, Social and Governance.} scores from the \textit{Thomson Reuters ASSET4} database, accessed through \textit{Datastream}. These scores are at the yearly frequency and available for a sub-sample of firms until end of year 2019. Similarly to accounting information, end of year scores are matched to daily observations of the subsequent year. When 2019 scores are missing, daily observations for 2020 are matched to the end of 2018 score. The main score included in the analysis is the \textit{E-score} that only takes into account firms' environmental outcomes. Results using the comprehensive ESG score are similar and available upon request.  

Merging all these data sets we end up with a panel of $1,821,974$ daily observations for slightly less than $4,000$ firms.

Great part of the analysis will focus on investor reaction to the market turbulence stirred by the Covid-19 outbreak. For this reason, we divide the sample in three main periods. The "Pre-pandemic" period ranges from May 2018 to February 21st 2020, the day of the last S\&P-500 peak preceding the market collapse. The "Crash" period follows, from February 21st until March 20th, when market valuations declined. Finally, the "Recovery" period begins on March 21st, just before the Federal Reserve's intervention, and reaches the end of the sample on August 7th 2020.

Table \ref{sumstats} shows summary statistics for the main variables as of the beginning of the crash period, on February 21st 2020. Similar tables for March 20th and August 7th can be found in the Appendix. [EXPAND ON TABLE COMMENT HERE !!!!!!!!]

\begin{table}
\centering
\caption{test} \label{sumstats}
\input{../../output/tables/sumtable1} 
\end{table}



To illustrate the growth in popularity of the Robinhood platform, we plot in Figure \ref{linevalue} an estimate of the total value of the stock portfolio in our sample over time. The estimate is computed assuming that each user holding a stock holds exactly one share. That is, we plot the number of users times the share price each day. This is a sizable underestimation, but is useful to describe the changes in the total portfolio value over time. In addition, this chart helps us visualize the sub-periods we highlighted before: "pre-pandemic" in white, "crash" in red and "recovery" in green. The main empirical analysis will aim at describing the cross section of stock popularity across these different periods.

\begin{figure}[h!]
\caption{Test}
\centering \label{linevalue}
\includegraphics[scale=0.5]{../../output/charts/value.png}
\end{figure}


\section{Main Empirical Analysis} \label{results}

We are after the determinants of stock popularity among retail investors, and how these changed during the Covid-19 crisis. The analysis will be carried out by means of cross-sectional regressions. First, in subsection \ref{pre}, we describe the cross-section of stock popularity during the pre-pandemic period. Subsequently, in subsection \ref{covid} we focus on changes in stock popularity during the crash and recovery periods.

\subsection{Cross-section of popularity pre-pandemic} \label{pre}

Table \ref{prereg} presents the results of the OLS estimation of a regression model where the dependent variable is the number of users holding a stock, and the explanatory variables are stock-level characteristics. In mathematical notation, the specification is:
$$
log(\# \hspace{1mm} Users_{i,j,t}) = \alpha_t + \sigma_j + \beta \mathbb{X}_{i,t} + \varepsilon_{i,j,t}
$$
where $i$, $j$ and $t$ indicate firm, sector and calendar day, respectively. The sample period ranges from May 5th 2018 until February 21st 2020.

The results are not surprising. Most coefficients indicate that Robinhood investors are more likely to hold larger stocks, with high beta and book-to-market ratio. Greener assets are also significantly more popular. Interestingly, profitability is negatively correlated with stock popularity. Past stock returns (momentum variable) do not seem to be a significant predictor of popularity.

\begin{table} 
\caption{Cross-Section of Stock Popularity Pre-Pandemic}
\input{../../output/tables/pre_fe_tab_enscore_slides} \label{prereg}
\end{table}




\subsection{Cross-section of popularity changes during the Covid crisis} \label{covid}

So far we determined which stock characteristics explain the distribution of users across stocks during "normal times". It is reasonable to believe that these characteristics may change once turbulent times arrive. The recent Covid-19 crisis represents an ideal scenario to test this conjecture. This section presents the results of a cross-sectional analysis similar to the one of section \ref{pre}, with two main differences. The first one is that now the dependent variable is the \emph{change} in the number of users holding each stock, instead of the level. This is to emphasize our focus on the reaction of investors to the shock: how do they reallocate capital across assets? The second obvious difference is the sample period considered. We will now home in on the two Covid sub-samples previously described: the "crash" period (Feb 21st-Mar 20th) and the "recovery" period (Mar 20th-Aug 7th).

The Covid-19 outbreak has clearly been an asymmetric shock, hitting some industries and firms considerably harder than others. In fact, for some companies, the recent months have been an unprecedented growth opportunity and spur for innovation. The question is to which stocks retail investors resorted during the market collapse and subsequent rebound.

Figures \ref{chart-established}, \ref{chart-winners} and \ref{chart-dead} display the users and share price progression since February for well-known firms chosen as examples. Figure \ref{chart-established} shows two large, established firms such as Microsoft and Google. From a business standpoint, both weathered the crisis well. At the same time, the number of Robinhood users with long positions on their stock experienced a constant increase. Other firms, such as Amazon and Tesla, depicted in Figure \ref{chart-winners}, were clear pandemic winners. Their stock price boomed alongside interest on the part of Robinhood traders. The most interesting pair of charts is however in Figure \ref{chart-dead}. These are firms that instead were hit the hardest by the pandemic, such as \textit{American Airlines}, the airline, and \textit{Hertz}, the car rental company. While their stock price collapsed, and in the case of Hertz the company was filing for bankruptcy, their popularity among Robinhood experienced an extraordinary increase. This may suggest that these investors were searching for inexpensive bargains, and were willing to bet on the resurrection of heavily battered, risky firms.



\begin{figure}
\centering
\caption{Test}
\begin{subfigure}{.5\textwidth}
\centering
\includegraphics[width=\linewidth]{../../output/charts/MSFT.png} 
\end{subfigure}% this percentage is important for some reason
\begin{subfigure}{.5\textwidth}
\centering
\includegraphics[width=\linewidth]{../../output/charts/GOOGL.png}
\end{subfigure}
\label{chart-established}
\end{figure}

\begin{figure}
\centering
\caption{Test}
\begin{subfigure}{.5\textwidth}
\centering
\includegraphics[width=\linewidth]{../../output/charts/AMZN.png} 
\end{subfigure}% this percentage is important for some reason
\begin{subfigure}{.5\textwidth}
\centering
\includegraphics[width=\linewidth]{../../output/charts/TSLA.png}
\end{subfigure}
\label{chart-winners}
\end{figure}

\begin{figure}
\centering
\caption{Test}
\begin{subfigure}{.5\textwidth}
\centering
\includegraphics[width=\linewidth]{../../output/charts/AAL.png} 
\end{subfigure}% this percentage is important for some reason
\begin{subfigure}{.5\textwidth}
\centering
\includegraphics[width=\linewidth]{../../output/charts/HTZ.png}
\end{subfigure}
\label{chart-dead}
\end{figure}

Some interesting patterns emerge also by looking at industries instead of single firms. Figure \ref{chart-sectors} displays the sectors that saw the highest and lowest percentage change in Robinhood users during the crash and recovery periods. In both periods, the largest green bars belong to sectors that suffered the crisis severely, such as airlines and hospitality. Especially during the recovery period, it seems that some of the sectors that were clear "pandemic winners" were also among those that lost the most in terms of Robinhood popularity. For instance, this is the case for Wireless telecommunication and Pharmaceuticals. This pattern is consistent with the conjecture that these are investors with a strong appetite for bargains, potentially at the expense of the solidity of firm fundamentals.

\begin{figure}
\centering
\caption{Test}
\begin{subfigure}{.8\textwidth}
\centering
\includegraphics[width = \linewidth]{../../output/charts/sectors_crash.png} 
\end{subfigure} 
\begin{subfigure}{.8\textwidth}
\centering
\includegraphics[width = \linewidth]{../../output/charts/sectors_recovery.png}
\end{subfigure}
\label{chart-sectors}
\end{figure}

We now move on to a more formal regression analysis of the determinants of cross-sectional changes in users during the crisis. First, we focus on changes in popularity during the crash period, from February 21st to March 20th 2020.
The specification we run is 
$$
log(\# \hspace{1mm} Users_{i,j,t+1})-log(\# \hspace{1mm} Users_{i,j,t}) = \alpha log(\# \hspace{1mm} Users_{i,j,t}) + \sigma_j + \beta \mathbb{X}_{i,t} + \varepsilon_{i,j,t}
$$
where date $t$ is February 21st and $t+1$ is March 20th, and $i$, $j$ and $t$ indicate firm, sector and calendar day, respectively. $\mathbb{X}_{i,t}$ is a vector of firm characteristics, and  $\sigma_j$ indicates sector fixed effects, where sectors are defined by 6-digit \textit{GICS} codes.  The results of the OLS estimation are reported in Table \ref{crashreg}. Focusing on the most complete specification, we observe that popularity changes are correlated with lower firm quality. Investors allocated capital towards firms with lower cash reserves, higher leverage, high market beta and lower profitability. Interestingly, momentum, defined as the average daily excess return over the month preceding February 21st, has a sizable negative coefficient. These investors display a behavior that is not consistent with a "flight to quality" during the market downturn. There is no evidence of investors resorting strongly to "green" assets either. The coefficient on the environmental score is positive and significant, but quite small in magnitude. This somewhat surprising given that this is a sample of young investors that should be more sensitive to environmental issues.

\begin{table}[H]
\caption{Cross-Section of Stock Popularity during Crash}
\input{../../output/tables/crash_fe_tab_enscore_slides} \label{crashreg}
\end{table}

In what follows, we run a similar test, now focusing on the "recovery" period, which coincides with the market improvement that took place since the intervention of the Federal Reserve in late March. In particular, we study changes in popularity from March 20th 2020 to August 7th 2020.
The specification we run is identical to the one of Table \ref{crashreg}, i.e.,
$$
log(\# \hspace{1mm} Users_{i,j,t+1})-log(\# \hspace{1mm} Users_{i,j,t}) = \alpha log(\# \hspace{1mm} Users_{i,j,t}) + \sigma_j + \beta \mathbb{X}_{i,t} + \varepsilon_{i,j,t}
$$
where date $t$ is March 20th and $t+1$ is August 7th, and $i$, $j$ and $t$ indicate firm, sector and calendar day, respectively. $\mathbb{X}_{i,t}$ is a vector of firm characteristics, and  $\sigma_j$ indicates sector fixed effects, where sectors are defined by 6-digit \textit{GICS} codes.  The results of the OLS estimation are reported in Table \ref{recoveryreg}. Let us focus on the rightmost column 5. Comparing the results with those of Table \ref{crashreg}, we notice that the main effects appear similar: high leverage, low cash and profitability predict increasing popularity. In this period however, also size and book-to-market are positively correlated with investments, and market beta is now a negative predictor of users change\footnote{Market beta for every firm is calculated using data from the previous year, 2019 in this case. It may therefore be the case that after the pandemic this measure does no reflect market co-variation properly.}. Moreover, good environmental measures do correlate with investors' interest during this period, perhaps because investors expect the future recovery of the economy to proceed in parallel with a green transition. Arguably, the most striking result is the coefficient on momentum. The momentum variable is the average daily excess return approximately during the crash period: the month leading up to March 20th. The coefficient is strongly significant, as it was in Table \ref{crashreg}, and here considerably larger in magnitude. A one percent higher momentum return leads to an eight percentage point lower change in users over the period considered. This result is consonant with the idea that Robinhood investors were in quest of stocks severely struck by the onset of the pandemic, potentially representing good value for money.


\begin{table}[H]
\caption{Cross-Section of Stock Popularity during Recovery}
\input{../../output/tables/recovery_fe_tab_enscore_slides} \label{recoveryreg}
\end{table}


\section{Conclusion} \label{conclusion}


\newpage
\nocite{*}
\bibliography{bibliography_2yp.bib} 
\bibliographystyle{plainnat}
 
 % APPENDIX
 
 \newpage
\input{../../output/tables/sumtable2}
\input{../../output/tables/sumtable3}
 
 [need one table with variable definitions !!!!!!!!!!!!!!!!!!!!!!!!]
 
\end{document}