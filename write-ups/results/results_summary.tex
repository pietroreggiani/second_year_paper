\documentclass[12pt]{article}
 
\usepackage[margin=1in]{geometry} 
\usepackage{amsmath,amsthm,amssymb, float}
\usepackage{listings}

\usepackage[utf8]{inputenc}
\usepackage{amsfonts}
\usepackage{graphicx}
%\usepackage{subcaption} alternative to subfig
\usepackage[round]{natbib}
\usepackage[english]{babel}
\usepackage{marginnote}
\usepackage{array}%for tables
\usepackage{hyperref}

\usepackage{subcaption}

\usepackage{pgffor}

    

%\usepackage{subfig}  %to align two figures side by side
\usepackage{float}
\usepackage[font={footnotesize}]{caption}  %to customize captions
\usepackage[noorphans,font=itshape]{quoting}


\numberwithin{equation}{section} %gets equation numbering follow the section
\linespread{1.3}

\graphicspath{{D:/Piter USB/PhD local folder/second_year_paper/outputs/charts/}}
\setlength{\parindent}{1cm}
\setlength{\parskip}{0.5cm plus4mm minus3mm}


\AtBeginDocument{\renewcommand{\bibname}{References}} %to call bibliography references
 
\newcommand{\N}{\mathbb{N}}
\newcommand{\Z}{\mathbb{Z}}
 
 
\begin{document}
 
% --------------------------------------------------------------
%                         Start here
% --------------------------------------------------------------
 
Pietro Reggiani August 2020



\section{Sin and Fossil Stocks Historical Alphas}
This section follows Hong and Kacperkzyk 2009 \textit{Price of Sin} paper. What they document is that stocks that belong to the sin category (alcohol tobacco and gaming) show positive abnormal returns, lower institutional ownership and lower analyst coverage.


 The goal of my exercise is to check the robustness of abnormal returns outside of the sample they use, and whether the fossil fuel industry displays any similar patterns. 
 
 I take monthly stock prices from CRSP, and monthly factor returns from Ken French's website. The time period considered is January 1980 to December 2019.
Within the CRSP universe, I isolate two different sub-samples of stocks. The first sub-sample is the one of "Sin Stocks" as defined by Hong and Kacperkzyk: beer and liquor, smoke and tobacco (industry categories follow SIC and NAICS codes). The second one is fossil stocks (coal, petroleum and natural gas SICs).
Then, I compute the monthly equal-weighted return for each sub-sample (fossil stock returns also have an alternative measure coming from the return in the Ken French industry portfolio file), and compare it to what is predicted by a standard four-factor model (FF 3-factor + momentum). 


Following Hong and Kacperkzyk, I run time series regressions of the type:
$$ Portfolio\_return_t = \alpha + \beta' * Factors_t + \varepsilon_t   $$
where \textit{Factors} represents a vector of four classic factors (market, size, value and momentum). The regressands are, in turn: 
\begin{itemize}
\item $excomp$ is the equal weighted monthly return of the sin portfolio minus the return of a portfolio of comparable industries as defined by Hong and Kacperkzyk (Food, Soda, Fun and Meals-restaurants-hotels in the Fama-French classification)
\item $sinret$ is the equal weighted monthly return of the sin portfolio
\item $fossilret$ is the equal weighted monthly return of CRSP stocks in coal, petroleum and natural gas sectors.
\item $fossil$ is the equal weighted monthly return of the Fama-French Coal and Oil portfolios together.
\end{itemize}

Hong and Kacperkzyk estimate $\alpha$ using 3-year rolling regressions and then averaging the coefficients to get a single estimate. I could not understand how to handle standard errors in this case, so I opted for estimating the alpha separately for different time periods. The estimates shown below are for 5 sub-samples of 6 years each (72 observations per sample). Confidence intervals at the 95\% confidence level are plotted around the estimates.

\foreach \n in {excomp, sinret, fossil, fossilret}{
\begin{figure}[H]
\includegraphics[scale=0.7]{\n_alphas.png}
\caption{The blue dots are the point estimates for the alpha of the time series regression. The bars represent 95\% confidence intervals. $excomp$ is the return on the sin portfolio minus a portfolio of "comparable" sectors (Food, Soda, Fun and Meals-restaurants-hotels in the Fama-French classification) $fossilret$ are returns of the coal and oil portfolio that I select from CRSP using SIC codes, while $fossil$ is the equal-weighted average of returns from the Fama-French coal and oil industry portfolios.}
\end{figure}
}

If instead I simply estimate $\alpha$ considering the whole sample 1980-2019, the estimates look like this.

 % latex table generated in R 4.0.2 by xtable 1.8-4 package
% Mon Aug 24 13:13:39 2020
\begin{table}[ht]
\centering
\begin{tabular}{rrrrrrr}
  \hline
 & sample & alpha & se & ttest & cint1 & cint2 \\ 
  \hline
1 & 18261.00 & 0.01 & 0.19 & -1.88 & -0.35 & 0.38 \\ 
  2 & 18261.00 & 0.66 & 0.19 & 1.51 & 0.29 & 1.04 \\ 
  3 & 18261.00 & -0.14 & 0.34 & -1.54 & -0.81 & 0.52 \\ 
  4 & 18261.00 & -0.02 & 0.30 & -1.89 & -0.61 & 0.57 \\ 
   \hline
\end{tabular}
\end{table}

 
Two things strike me as potentially interesting. The consistent over-performance of the sin portfolio (albeit not when considered in excess of comparables), and the under-performance of the fossil fuel sector in the recent period. This under-performance is robust to splitting the sample in even smaller sub-periods, and is consistent with a shift away from brown firms in the recent period (mind that the covid months, extremely bad for the fossil industry, are not part of the sample).
 

 
 
\end{document}